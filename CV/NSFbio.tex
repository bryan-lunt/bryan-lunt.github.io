
%%%%%%%%% BIOGRAPHICAL SKETCH -- 2 pages

\required{Biographical Sketch: Bryan James Lunt}
% Your Bio should be divided into the following sections
% (a) Professional Preparation (education):
% Undergrad, Major, Year
% Graduate, Major, Year
% Postdoc, Area, Years-Inclusive
% (b) Appointments:  most recent first.
% (c) Publications:  5 related to the proposal, and 5 "Other Significant Publications"
% (d) Synergistic Activities (math-enhancing activities that were not
% part of your main job description, like editorial boards and
% conference organizing - any Math-related volunteer work.
% these are often similar to Broader Impacts
%
% (e) Collaborators & Other Affiliations: (use the following sections)
% list in alphabetical order, and include current affiliations parenthetically
% Collaborators and Co-editors: past 48 months.  If none, write "none"

Bryan Lunt is currently in the second year of a Master of Science degree in Computer Science at the University of California, San Diego under the tutelage of professor Charles Elkan.
Of note for this application is his focused research in the particular field of Protein-Protein Interaction and Protein Contact Map Prediction.

\nocite{Morcos2011a,Procaccini2011a,Lunt2010a}
\section*{Professional Preparation}
\begin{tabular}{c c c}
UC, San Diego & {\bf M.S. Computer Science} & Sep. 2011 - June 2013 {\it (in progress)}\\
CSU, Chico & {\bf Continued Education: Physics and CS} & Jan. - May 2011\\
CSU, Chico & {\bf B.S. Computer Science}  & {\it Graduated} May 2008\\
\end{tabular}

\section*{Appointments}
\begin{itemize}
	\item {\bf Teaching Assistant}; Computer Science and Engineering; UC San Diego; Sep.-Dec. 2012 {\it (Current)}
		\subitem Supervisor: {\bf Professor Charles Elkan }
		\subitem Assistant for {\it CSE 250A: Probabilistic Reasoning and Learning}: Grading quizzes and homeworks, holding office hours, leading review sections.
		
	\item {\bf Research Assistant}; Computer Science and Engineering; UC San Diego; Mar. 2012 - {\it (Current)}
		\subitem Supervisor: {\bf Professor Charles Elkan }
		\subitem Researching improvements into the method of \cite{Morcos2011a} and synergistic combinations of those methods with methods from Machine Learning.
	
	\item {\bf Research Assistant}; Center for Theoretical Biological Physics; UC San Diego; Dec. 2009 - Oct. 2010
		\subitem Supervisor: {\bf Professor Terence Hwa }
		\subitem Used methods developed below to investigate the bacterial Tryptophan biosynthesis pathway, and predict multimerization interfaces for eukaryotic cytoskeletal proteins. Created webservice versions of all major tools for covariance analysis. Contributed heavily to \cite{Morcos2011a}.
	
	\item {\bf Visiting Scientist}; Max Planck Institute for Molecular Genetics; Berlin, Germany; Sep - Nov. 2009
		\subitem Supervisor: {\bf Group Leader Peter Arndt }
		\subitem Investigated the dynamics of ancestor reconstruction under various phylogenetic and mutational models.
	
	\item {\bf Research Assistant}; Institute for Scientific Interchange Foundation; Torino, Italy; Jan. - July 2009
		\subitem Supervisor: {\bf Professor Martin Weigt }
		\subitem Continued development of tools for Protein-Protein interaction prediction. Developing standardized user-friendly dataset generation tools. Published \cite{Lunt2010a}, and contributed to \cite{Procaccini2011a}.
	
	\item {\bf Research Assistant}; Center for Theoretical Biological Physics; UC San Diego; June - Dec. 2008
		\subitem Supervisor: {\bf Professor Terence Hwa }
		\subitem Developed software and databases for Protein-Protein interaction prediction using information theoretic methods, and tools for creating nightly sequence alignments based on latest available genome data. Performed systems administrative tasks.
		
	\item {\bf Japanese-English Translator / Software Developer}; Musketeer Information Research; Chico, CA / Yokohama, Japan; Feb. 2006 - May. 2008
		\subitem Supervisor: {\bf Self / Client }
		\subitem Developed mission-critical applications, set IT policy, provided remote and on-site disaster-response. Rapidly prototyped software to support newly developing products and workflows in an environment of constantly changing requirements. Carefully interviewed client staff to educe requirements. Summarised and translated Japanese financial news articles, identifying and focusing on points interesting to the client, points likely to draw reader attention, and important figures. Created coherent summaries that flow smoothly.

\end{itemize}

\section*{Publications}
\nocite{Morcos2011a,Procaccini2011a,Lunt2010a}
\renewcommand{\refname}{\vspace*{-1cm}}
\bibliography{BryanPubs}

\section*{Collaborators and Other Affiliations}
%%\subsection*{Influential Collborators}
Among the co-authors listed above, Bryan Lunt maintains ongoing scientific contact and collaboration with the following:
\begin{itemize}
%\item {\bf Arianna Bertolino } Fondazione Istituto per l'Interscambio Scientifico; Turino, Italy
%\item {\bf Professor James A. Hoch } Department of Molecular and Experimental Medicine ; The Scripps Research Institute
\item {\bf Professor Terence Hwa } Center for Theoretical Biological Physics; UC San Diego
%\item {\bf Debora S. Marks } Department of Systems Biology, Harvard Medical School
\item {\bf Faruck Morcos } Center for Theoretical Biological Physics; Rice University
%\item {\bf Professor Jos\'{e} N Onuchic } Center for Theoretical Biological Physics; Rice University
\item {\bf Andrea Pagnani } Fondazione Istituto per l'Interscambio Scientifico; Turino, Italy
%\item {\bf Andrea Procaccini } Fondazione Istituto per l'Interscambio Scientifico; Turino, Italy
%\item {\bf Professor Chris Sander } Computational Biology Program; Memorial Sloan-Kettering Cancer Center
%\item {\bf Professor Hendrik Szurmant } Department of Molecular and Experimental Medicine; The Scripps Research Institute
\item {\bf Professor Martin Weigt } Laboratoire de G\'{e}nomique des Microorganismes; Universit\'{e} Pierre et Marie Curie
%\item {\bf Professor Riccardo Zecchina } Theoretical Physics; Department of Applied Science and Technology; Politecnico di Torino
\end{itemize}

\subsection*{M.S. Thesis Advisor: Professor Charles Elkan}
Department of Computer Science and Engineering; University of California, San Diego
%\subsection*{Professional Affiliations}
%\begin{itemize}
%\item {\bf International Society for Computational Biology} July 2012 - {\it (ongoing)}
%\end{itemize}


